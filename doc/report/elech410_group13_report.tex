\documentclass[a4paper, twoside, 12pt]{article}
%packages
\usepackage[utf8]{inputenc}
\usepackage{graphicx}
\usepackage[]{background}
\usepackage[a4paper,top=2cm,bottom=2cm,left=2cm,right=2cm,marginparwidth=2cm]{geometry}
\usepackage[english]{babel}
\usepackage[section]{placeins}
\usepackage{enumitem}
\usepackage{float}
\usepackage[colorinlistoftodos]{todonotes}
\usepackage{multirow}
\usepackage{amsmath}
\usepackage{mathrsfs}
\usepackage{eurosym}
\usepackage{fancyhdr}
\pagestyle{fancy}
\usepackage{titlesec}
\usepackage{caption}
\usepackage{wrapfig}
\usepackage{tabularx}
\usepackage[table]{xcolor}
\usepackage{pgfplots}
\pgfplotsset{compat=1.17}
\usepackage[backend=biber, style=numeric]{biblatex}
\usepackage[colorlinks=true, linkcolor=blue, citecolor=blue]{hyperref}
\usetikzlibrary{shapes.geometric, arrows}


\linespread{1}


\renewcommand{\headrulewidth}{2pt}
\renewcommand{\sectionmark}[1]{\markboth{#1}{#1}}
\fancyhead[R]{}
\fancyhead[L]{\textit{\leftmark}}


%\renewcommand{\familydefault}{\sfdefault}

%
\begin{document}
\begin{titlepage}

	\newcommand{\HRule}{\rule{\linewidth}{0.5mm}} % Defines a new command for the horizontal lines, change thickness here

	%----------------------------------------------------------------------------------------
	%	LOGO SECTION
	%----------------------------------------------------------------------------------------

	\includegraphics[width=8cm]{img/logo-polytech.jpg}\\[1cm] % Include a department/university logo - this will require the graphicx package

	%----------------------------------------------------------------------------------------

	\center % Center everything on the page

	%----------------------------------------------------------------------------------------
	%	HEADING SECTIONS
	%----------------------------------------------------------------------------------------

	\textsc{\LARGE ELEC-H410}\\[1.5cm] % Name of your university/college
	\textsc{\Large Université Libre de Bruxelles}\\[0.5cm] % Major heading such as course name
	\textsc{\large École Polytechnique de Bruxelles}\\[0.5cm] % Minor heading such as course title

	%----------------------------------------------------------------------------------------
	%	TITLE SECTION
	%----------------------------------------------------------------------------------------
	\makeatletter
	\HRule \\[0.4cm]
	{ \huge \bfseries Pandemic Project}\\[0.4cm] % Title of your document
	\HRule \\[1.5cm]

	%----------------------------------------------------------------------------------------
	%	AUTHOR SECTION
	%----------------------------------------------------------------------------------------

	\begin{minipage}{0.4\textwidth}
		\begin{flushleft} \large
			\emph{Authors of Group 13:}\\ % Your name

			JANKE Nico (540076) \\ VAN Dyck Emile\\ WOJTACH Kacper (513025)
		\end{flushleft}
	\end{minipage}
	~
	\begin{minipage}{0.4\textwidth}
		\begin{flushright} \large
			\emph{Professor:} \\
			Quitin François
		\end{flushright}
	\end{minipage}\\[2cm]
	\makeatother

	% If you don't want a supervisor, uncomment the two lines below and remove the section above
	%\Large \emph{Author:}\\
	%John \textsc{Smith}\\[3cm] % Your name

	%----------------------------------------------------------------------------------------
	%	DATE SECTION
	%----------------------------------------------------------------------------------------

	{\large \today}\\[2cm] % Date, change the \today to a set date if you want to be precise

	\backgroundsetup{scale=1,
		color=black,
		opacity=0.17,
		angle=10,
		position={12cm,-22cm},contents={%
				\includegraphics[height=20cm,width=20cm,keepaspectratio]{img/sceauULB.jpg}}%
	}

	\vfill % Fill the rest of the page with whitespace

\end{titlepage}

\newpage
\BgThispage
\backgroundsetup{contents={}}
\tableofcontents
\newpage

\section{Introduction}
The goal of the pandemic game is to produce enough vaccines to eradicate the virus
before the population collapses. We have at our disposal an interface (\textit{pandemic.h})
to communicate with the game routine.\\
\begin{itemize}
	\item Every 3 seconds a vaccine clue is dropped.
	\item Every 5 seconds the percentage of healthy people in the population is updated.
	\item On random occasions an individual gets contaminated and we need to quarantine
	      the population in less than 10 milliseconds.
\end{itemize}
To manage these events and win the game we created one task for each with different
priorities. We also have a task dedicated to printing the game state on a LCD screen.

\section{Tasks and Synchronization}
\subsection{Quarantine}
This task has the highest priority of all our tasks. All it does is wait for the
\textit{QuarantineStart} binary semaphore to be available. This is triggered by the contamination.
When the semaphore is available the task starts the quarantine. This is done in time because
of the priority of the task.

\subsection{Vaccine}
The \textit{vaccineProductionTask} has the second highest priority. It waits for a clue to be
released to start producing a vaccine. That way we start producing as soon as we get a
clue. This is needed because we have 3 seconds to produce a vaccine after the release.\\
When the vaccine is produced we ship it immediately. Once this is done we give the lcd
and lab semaphores. The first one is to notify the LCD task that the game state has changed
and the second one is to notify the Medicine task that it can producce or ship a medicine.

\subsection{Medicine}
As specified is the previous section, the \textit{medicineProductionTask} is notified by
the vaccine task when a vaccine is shipped and there is time to produce or ship a medicine.
This task because of its lower priority and the fact that it has to wait for the vaccine task
only runs when there is time between vaccine production.\\
This will be more visible in the logic analyzer traces.

\subsection{LCD}
This last task is the lowest priority one. It is notified by the vaccine and medicine tasks
when the game state has changed. It will print the game state as specified in the
specification.

\section{Logic Analyzer Traces}

\section{Conclusion}

\nocite{*}
\end{document}
